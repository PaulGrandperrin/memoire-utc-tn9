\part{Déroulement du stage}
\chapter{Contexte}
\section{Le cloud computing\index{cloud computing}}
\subsection{Principe général}
\paragraph*{}
Le \emph{cloud computing} \footnote{Litéralement: "informatique dans les nuages"} est expression
très à la mode ces dernières années et possède une définission très large.
Le principe général est d'externaliser sur des serveurs distants des données ou
des traitements informatiques.

\paragraph*{}
L'avantage pour l'utilisateur du service est qu'il n'a plus à ce soucier d'où se situent physiquement
ses données ni de l'administration des serveurs. Toutes les contraintes de sécurité, de disponibilité
et de fiabilité sont déportées à la responsabilité de l'hébergeur.

\paragraph*{}
Pour l'hébergeur, les avantages sont aussi certain: Le cloud computing permet de mutualiser au maximum
les infrastructures matériels et d'industrialiser une grande partie des processus d'administration.
Par conséquence cette technique permet de diminuer sensiblement les coûts et/ou d'augmenter la qualité
de service.

\subsection{Le cloud appliqué à l'hébergement de services ou d'OS\protect\footnote{De l'anglais \textit{Operating System}: Système d'exploitation}}
\paragraph*{}
Alter Way Hosting ...


\begin{pygmented}{c}
void main(int argc, char* argv[])
{
	printf("hello");
	return 0;
}
\end{pygmented}

\cite{test}
