\part{Déroulement du stage}

\paragraph*{}
\lipsum[1]


\chapter{Modification d'OpenNebula}

\paragraph*{}
Les premiers jours de mon stage ont été dédiés pour tester les logiciels OpenNebula et Xen sur une platforme de test
composée d'un petit serveur accueillant OpenNebula, de deux machines utilisées comme hyperviseurs Xen et d'un NetApp.
Vivien, pour responsable de stage, avait déjà déployé un environnement de développement fonctionnel sur ces machines.

\paragraph*{}
Afin de prendre en main la platforme, mon premier objectif à été de mettre à jour OpenNebula qui était en version 2.2 vers la version 3.0 Beta 1.
Cette version 2.2 a été modifié par Vivien pour fonctionner avec un plugin qu'il a écrit pour ajouter le support de l'iSCSI pour déporter le stockages des images
disque des VMs sur le NetApp.
\\
Il m'a donc fallu comprendre puis adapter le travail de Vivien. J'ai ensuite effectué beaucoup de tests pour valider le fonctionnement
de la nouvelle solution et aussi en valider ma compréhension.
\\
Après avoir compris comment marchait OpenNebula de l'extérieur - en tant qu'utilisateur - j'ai téléchargé les sources et ai préparé un
environnement de développement pour éditer, compiler\footnote{Étape de construction de l'exécutable du logiciel à partir des sources},
débugger\footnote{Action de compiler et lancer un exécutable d'une certaine manière permettant de voir le fonctionnement interne de l'application
pendant son fonctionnement pour comprendre et corriger des bugs}, installer et tester le logiciel.


\paragraph*{}
Pour l'ajout des fonctions de \emph{scale-in}, j'ai du en même temps regarder ce qui était techniquement possible au niveau de l'hyperviseur Xen qui
sera utilisé comme backend\footnote{Un backend est la partie basse d'une architecture utilisé pour éffectuer des actions à l'extérieur du logiciel.
Par opposition le frontend est la partie haute de l'architecture recevant les ordres de l'utilisateur et commandant les autres parties de l'architecture}
par OpenNebula pour crééer les VMs sur un hyperviseur.

\paragraph*{}
En effet, en fonction des différentes version du noyau Linux et de Xen, certaines fonctionnalités sont disponibles, indisponibles ou buggés.\\
J'ai donc fait un tableau LibreOffice Calc\footnote{Logiciel open source concurrent de Microsoft\textsuperscript{\textregistered} Excel\texttrademark}
pour répertorier l'état de chaques fonctionnalités en fonction de toutes les combinaisons de versions possibles.
\\
Grâce à ce tableau, nous avons pu déterminer quelles étaient les fonctionnalités nous utiliserons et par déduction quelles versions de logiciels nous
installerons.

\paragraph*{}
Après avoir expérimenté avec la base de code d'OpenNebula nous avons fait des dessins d'architecture et une déterminé une procedure pour mettre en oeuvre ces
modifications.
\\
L'objectif était de commencer par s'occuper des modifications de la gestion du stockage puis d'ajouter le support du \emph{scale in}.

\section{La gestion du stackage des images disque}
\paragraph*{}
Pour l'instant OpenNebula considère que les images disque des VMs sont soit stockées en local sur l'hyperviseur soit partagées sur un répertoire NFS
	\footnote{NFS: Network File System, est un système de fichier qui permet d'accèder à des fichiers via le réseau. Plusieurs clients peuvent accèder
	au mêmes fichiers en même temps. Ce système ressemble au protocole SMB de Microsoft\textsuperscript{\textregistered}}\index{NFS}
.
Vivien à aussi écrit un plugin OpenNebula pour partager les images disque via l'iSCSI - plus performant que le NFS pour ce type d'utilisation.
\\
Dans tout les cas, OpenNebula n'a pas été designé pour gérer plusieurs serveurs de stockage et répartir les images disque dessus.
\\
Les deux objets les plus important manipulés par OpenNebula sont les \emph{Hosts} et les VMs. Un \emph{Host} est le nom utilisé par OpenNebula pour
désigner un hyperviseur.
Un \emph{Host} contient donc zéro ou plusieurs VMs et une VM a forcément un et un seul \emph{Host} associé.


\paragraph*{}
Notre objectif serait de faire la même relation entre les images disques vs les serveurs de stockage et les VMs vs les \emph{Hosts}.
\\
OpenNebula n'a pour l'instant pas la notion de ce qu'est un serveur de stockage. Une image disque est juste un attribut d'une VM et n'a
pas de propriété faite pour désigner sur quel serveur de stockage elle se trouve.

\paragraph*{}
Nous avons donc créée dans le code d'OpenNebula un nouveau type d'objet appelé \emph{storage backend}, littéralement << object de gestion du stockage >>.



\paragraph*{}


\begin{lstlisting}
oneadmin@opennebula:~$ onehost list
  ID NAME               RVM   TCPU   FCPU   ACPU   TMEM   FMEM   AMEM   STAT
   0 hyp1                 1    400    399    300     4G   2.7G   3.9G     on
   1 hyp2                 2    400    399    100     4G   1.2G   3.2G     on
\end{lstlisting}

\begin{lstlisting}
oneadmin@opennebula:~$ onevm list
ID USER     GROUP    NAME         STAT CPU     MEM        HOSTNAME        TIME
 0 oneadmin oneadmin vm1          runn   1    256M            hyp2 08 03:36:24
 1 oneadmin oneadmin vm2          runn   4   2048M            hyp2 00 00:01:38
 2 oneadmin oneadmin vm3          runn   2    128M            hyp1 00 00:01:00
\end{lstlisting}

\begin{pygmented}{c}
void main(int argc, char* argv[])
{
	printf("hello");
	return 0;
}
\end{pygmented}

\cite{test}
